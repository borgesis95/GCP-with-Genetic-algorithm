\documentclass{article}
\usepackage[utf8]{inputenc}
\usepackage{graphicx}
\usepackage{caption}
\usepackage{algpseudocode}
\usepackage{algorithm}
\usepackage{booktabs}
\usepackage{tabularx}
\usepackage{tabularray}
\title{Applicazione di un algoritmo genetico per la risoluzione del  Graph coloring problem}
\author{Salvatore Borgesi}
\date{Dicembre 2022/Gennaio 2023}

\begin{document}
\maketitle
\section{Introduzione}
Il problema da trattare consiste nel colorare i vertici di un grafo considerando un vincolo. Dato un numero di colori, vi è la necessità di colorare i vertici in maniera tale che due nodi adiacenti non abbiano lo stesso colore. 
Il numero minimo di colori che può essere utilizzato viene definito  \textbf{chromatic number} \chi(G). \end{chi} 
La risoluzione a questo problema può essere adottata in diversi contesti applicativi  quali:
\begin{enumerate}
    \item Scheduling.
    \item Gioco del Sudoku.
    \item Colorazione delle mappe.
\end{itemize}


\noindent La scelta dell'algoritmo genetico per il problema, è stata dettata dal fatto che essa mi consente di migliorare le soluzioni ottenute iterazione per iterazione, e mandando al successivo step,  solo le soluzioni che si "avvicinano" di più all'ottimo.

\section {Rappresentazione della popolazione iniziale}
Per l'inizializzazione della popolazione iniziale viene prima di tutto definito un vettore di numeri che va da 1 fino al numero massimo di colori che si vuole utilizzare per colorare i vertici del grafo affinchè venga rispettato il vincolo di colorazione del problema. Supponiamo dunque, che il numero di colori che si vuole utilizzare sia pari a n = 10. L'array che definisce i  colori sarà [1,2,3...,10]. Data quest'ultima lista, viene successivamente definita una soluzione (rappresentata anche essa da un'array) di lunghezza pari al numero di vertici del grafo. A ciascun cassetto della lista, viene asseganto in maniera casuale uno dei valori numerici definiti nell'arrray precedente. Questa operazione di generazione di un singolo individuo, viene eseguita ripetutamente fino a raggiungere il numero di individui desiderati (rappresentato in un file di configurazione tramite la variabile POPULATON SIZE).

\begin{algorithm}
\caption{Initialize Population}\label{alg:cap}
\begin{algorithmic} 
\State \While{$Population \leq POPULATION\_SIZE$}
    \State {$solution \gets [ ]$}
    \State {$colors \gets [1 ... StartColorSize]$}
        \For{\texttt{each vertex}}
            \State{$color \gets random(colors)$}
            \State{$solution[vertex] \gets $ color$}
        \endfor

\endwhile
\end{algorithmic}
\end{algorithm}

\section{Funzione di fitness}
In un algoritmo genetico, la funzione di fitness consente di comprendere quale tra K individui è il migliore e quindi andrà con una certa probabilità alle iterazioni successive. La fitness definita all'interno della soluzione costruita, esamina ciascun arco del grafo, e per ciascuno viene verificato se due nodi adiacenti hanno lo stesso colore. In caso di esito positivo viene aggiunta una penalità. Successivamente, la somma degli archi che collegano due vertici aventi lo stesso colore, viene moltiplicata per il numero di colori che sta utilizzando la soluzione valutata. Questo prodotto perciò, restituisce zero se tutti i vertici sono colorati in maniera corretta. Dopo diversi esperimenti ho notato che questa prima "versione" di valutazione ad un determinato punto non riusciva più a migliorare poichè se l'algoritmo trovava due soluzioni con fitness pari a zero, non riusciva a capire quale delle due soluzioni era la migliore .Questo perchè, nonostante le soluzioni erano entrambe valide, una di esse potenzialmente poteva avere un valore cromatico più basso . Proprio per tale motivo ho deciso di aggiungere al prodotto citato prima, il numero di colori utilizzato. Facendo un esempio si supponga di avere due soluzioni corrette nella quale  la prima ha 5 colori invece la seconda ne ha 6 . Con la seconda funzione di fitness descritta, nonostante le due soluzioni siano entrambe corrette, la prima è migliore in quanto utilizza un minor numero di colori per il grafo.

\begin{algorithm}
\caption{Fitness function}\label{alg:cap}
\begin{algorithmic} 
    \State {$count \gets 0$}
    \State {$colors \gets numero\ di\ colori\ usati\ dalla\ soluzione$}
        \For{\texttt{each edges}}
            \If{$colore\ vertice\ u == colore\ vertice\ v $} 
            \State $count \gets count+1$
            \EndIf
        \EndFor
    \State \textbf{return} $(colors * count) + colors$}
\end{algorithmic}
\end{algorithm}

\section{Algoritmo utilizzato e pseudocodice}
In questa sezione della relazione scenderò più nel dettaglio nell'esaminare e descrivere l'algoritmo costruito per la risoluzione del problema.

\begin{algorithm}
\caption{Genetic algorithm}\label{alg:cap}
\begin{algorithmic} 
    \State{$graph \gets TranslateDimacsInstance()$}
    \State {$Population \gets InitializePopulation(graph)$}
    \State {$ColoreIniziale \gets Getupperbound(graph)$} 
    \State {$fitnessCount \gets 0$}
    \While {$ not\  Stopping \ criteria $}
    \State {$nuovaPopolazione\gets []$}
        \For{\texttt {i to POPULATION\_\ SIZE/2}}
            \State{$Selection();$}
            \State{$Crossover();$}
            \State{$Mutation();$}
            \State{$Replacement();$}

            \If{$FitnessAbsoluteBestSolution \geq ActualBestSolution$} 
                \State{$FitnessAbsoluteBestSolution \gets ActualBestSolution$}
            \EndIf
            \If{$fitnessCount > MAX\_\ NUMERO \_\ VALUTAZIONI$}
                \State{$StoppingCriteria \gets True$} 
\end{algorithmic}
\end{algorithm}



\noindent Come descritto  sopra, la prima parte consiste nel tradurre le istanze fornite nel formato .col in maniera tale da poter essere rappresentate mediante codice. Per far questo sono state definite due classi : \textit{Graph()} e \textit{Vertex()}. All'interno dell'oggetto grafo saranno presenti la lista di archi e la lista di vertici. Ogni vertice tiene traccia di chi sono i nodi adiacienti ad esso.


\noindent Lo pseudocodice, mostra come le operazioni più importanti siano quelle di \textit{Selection}, \textit{Crossover}, \textit{Mutation} e di \textit{Replacement} che sono alla base del funzionamento di un algoritmo genetico. Dopo aver generato la popolazione iniziale, viene determinato mediante il metodo \textit{GetUpperbound()} qual'è il colore con il quale si inizerà a 'dipingere' i nodi del grafo. Questo metodo banalmente, prende come riferimento il numero massimo di archi uscenti per ciascun vertice e ne restituisce il valore più alto . L'algoritmo si conclude quando è stato raggiunto il numero massimo di valutazioni. Qual'ora una fitness è migliore di un altra (e la soluzione risulta valida) , viene sostituita la AbsoluteBestFitness precedente con quella appena trovata.

\subsection{Selection}
Il processo di \textit{Selezione} consiste nel selezionare tra la popolazione generata nella fase di inizializzazione la "\textit{prole}" che sarà portata avanti nelle successive generazioni .In questo progetto sono state implementate tre tipi di operatori di selezione.

\subsubsection{Roulette}

Tramite questa modalità di selezione, vengono selezionati con probabilità più alta i genitori che hanno una fitness migliore. Per come è stata calcolata la fitness in questo progetto, i genitori che hanno un valore di fitness basso, avranno una maggiore probabilità di essere selezionati.

\subsubsection{Random}
Questa strategia prevede una selezione casuale dei genitori. 

\subsubsection{Tournament}
Con l'utilizzo di questa strategia vengono estratti K individui dalla popolazione. Di questi K individui, viene selezionato quello che il valore di fitness migliore.

\subsubsection{Strategia adottata per la selezione}
Mentre i tre metodi sopra citati sono stati implementati nel codice, la decisione finale è stata quella di utilizzare il la modalità \textit{Tournament} con un valore di K = 20. L'utilizzo di questo operatore infatti, consente anche a configurazioni non ottimali di fare parte delle future generazioni, dando l'opportunità di spaziare tra le varie soluzioni e non convergere dunque ad un ottimo locale. Sono stati effettuati diversi tentativi con K = 10. Con K = 10 infatti, ho notato che anche se l'algoritmo arriva a convergenza più velocemente, rischia dopo diverse iterazioni di rimanere bloccato in una ottimalità locale.

\section{Crossover}
L'operatore genetico di Crossover offre l'opportunità di mescolare le soluzioni ottenute mediante l'operatore di selezione ed ottenere dunque dei nuovi figli. Gli operatori di crossover sviluppati durante la realizzazione del prodotto sono il \textit{Single one-point crossover} e il \textit{Two-point crossover}.

\subsection{Single one-point crossover}
Dai figli generati durante la fase di Selezione, viene selezionato in modo tutto casuale un punto che ci consentirà di tagliare la lista in due parti.

\begin{center}
\includegraphics[width=0.5\textwidth]{images/one_point_crossover.png}
\end{center}
\subsection{Two-point crossover}
In questo caso, invece di selezionare un unico punto di taglio, vengono scelti due punti (randomici) delle due liste e vengono successivamente mischiate le due stringhe.

\begin{center}
\includegraphics[width=0.5\textwidth]{images/two_point_crossover.png}
\end{center}
\subsection{Stratgia adottata per il crossover}

Inizialmente, nella realizzazione del progetto ho utilizzato il single one-point crossover con una proabilità di 0.9. Ho notato tuttavia che utilizzare il 2-point crossover anche se con una probabilità più bassa (0.8) riesce ad apportare delle modifiche alle soluzioni generate e per tale motivo è stata mantenuta tale scelta per gli esperimenti. 


\section{Mutation}
L'operazione di mutazione , dato un cromosoma consente di variarne un 'gene', ottenendo come conseguenza una nuova soluzione. Per questo progetto sono state sviluppate due varianti.

\subsection{Random mutation}
Rappresenta l'operazione di mutazione più semplice. In breve, viene selezionato un elemento del cromosoma, viene selezionato un colore (in maniera casuale), e quest'ultimo (il colore) viene assegnato all'elemento scelto durante la prima fase.

\begin{algorithm}
\caption{Random mutation}\label{alg:cap}
\begin{algorithmic} 
    \State {$posizioneDaModificare \gets random(1,numeroVertici)$}
    \State {$ numeroColori \gets numeroColoriPresentiNellaSoluzione$}
    \State {$soluzione[posizioneDaModificare] \gets random(1,numeroColori)$} 
   
    
\end{algorithmic}
\end{algorithm}


\subsection{Vertex mutation}
Questo tipo di mutazione è stata costruita per risolvere il problema assegnato. 

\begin{algorithm}
\caption{Vertex mutation}\label{alg:cap}
\begin{algorithmic} 
    \State {$ColoriModificati \gets 0$} 
        \For{\texttt {vertex \textbf{in} Vertices}}
            \State{$neighbors \gets vertex.getNeighbors()$}
             \For{\texttt {neighbor in Neighbors}}
                \If{$solution[vertex] == solution[neighbor]$}
                    \State{$solution[vertex] \gets random(1,numeroColoriUsatiDallaSoluzione)$}
                    \State{$coloriModificati \gets coloriModificati + 1$}
                \EndIf
             \EndFor
        \EndFor
    \If{$ColoriModificati == 0$}
        \State{$posDaModificare  \gets random(1,numeroVertici)$}
        \State{$posCasuale  \gets random(1,numeroVertici)$}
          \If{$colorePosizioneDaModifcare == colorePosizioneCasuale$}
            \State{$colorePosizioneDaModificare \gets random(1,arrayColori)$}
          \Else
            \State{$colorePosizioneDaModificare \gets colorePosizioneCasuale$}
          \EndIf
\end{algorithmic}
\end{algorithm}


\noindent Per ciascun vertice, viene recuperata e ciclata la lista dei suoi nodi adiacenti. Dunque, per ciascun 'vicino' viene effettuato un controllo ovvero se il vertice e i vicini hanno lo stesso colore. In caso di esito positivo viene assegnato randomicamente un colore in un range che va da 1 al numero massimo di colori utilizzati in quell'istante dalla soluzione . Inoltre, viene incrementato un contatore che identifica il numero di colori che sono stati modificati. Dopo aver esaminato ciascun vertice ed il proprio vicinato, se risulta che durante le iterazioni non è stato modificato nessun colore, vuol dire che la soluzione è valida e di conseguenza si sceglierà in maniera randomica una la posizione della soluzione alla quale sarà assegnato un colore diverso. Questa seconda parte del codice è stata introdotta poichè senza di essa, dopo un discreto numero di iterazioni, gli individui ottenuti si fossilizzavano in un ottimo locale non andando a decrementare il numero di colori utilizzato.



\subsection{Strategie adottate per la Mutation}
Per alcune istanze , sono stati condotti degli esperimenti sia applicando la random mutation che la "Vertex mutation" sviluppata ad Hoc con alcuni accorgimenti legati alla rappresentazione del problema. 

%% GRAFICO

Il grafico in figura mostra un esempio di due iterazioni dove la linea (DI UN COLORE) rappresenta l'andamento del numero di colori in funzione delle valutaioni effettuate, utilizzando la random mutation, mentre la linea di colore (NON SO QUALE) mostra l'andamento per la vertex mutation. Si può facilmente notare che nel secondo caso il valore che rappresenta i colori converge più velocemente e inoltre raggiunge un valore più basso rispetto al primo caso. 

\section{Replacement}
Questa fase consente di sostituire la popolazione. Fondamentalmente sono state utilizzate due tipi di replacement.  La prima è la \begin \mu + \lambda \ replacement  \end   ..  Questa soluzione dapprima, mette insieme la popolazione attuale con quella appena generata. La nuova popolazione dunque sarà formata dagli individui  che presentano il valore di fitness più basso. Questo replacement avviene con una certa probabilità (impostata dal file di configurazione a 0.2). Altrimenti la popolazione appena costruita rimpiazzerà totalmente la popolazione precedente .

\section{Risultati ottenuti e conclusioni}

Sono stati realizzati due esperimenti ove la modifica più sostanziale è stato l'utilizzo nel primo caso della mutazione random, mentre nel secondo caso è stata utilizzata la Vertex mutation. In tabella vengono riportati alcuni dei valori del primo esperimento:






\begin{table}[htbp]
\begin{center}
\begin{tabular}{|l|l|l|l|l|} 
 \toprule Nome istanza  & Miglior colore & Media & Std  & Tempo di esecuzione \\ 
\midrule
queen5\_5.col  & 5              & 6.8   & 0.74 & 28m                 \\ 
queen6\_6.col  & 8              & 9.1   & 0.7  & 41m                 \\ 
queen7\_7.col  & 10             & 11.6  & 1.13 & 1h and 1m           \\ 
queen8\_8.col  & 14             & 14.2  & 0.39 & 1h and 30m          \\ 
queen8\_12.col & 18             & 19.3  & 0.89 & 3h and 12m          \\ 
queen9\_9.col & 15             & 17.1  & 1.04 & 2h and 14m          \\ 
\bottomrule
\end{tabular}
\caption{\label{table:tabella1}Risultati ottenuti con la random mutation}
\end{center}
\end{table}

\noindent I risultati della tabella ci danno alcune informazioni di cui vale la pena confrontare con la tabella \ref{table:tabella2}  che rappresenta invece gli esperimenti (di tutte le istanze assegnate) effettuati con la vertex mutation:


\begin{table}[htbp]
\begin{center}
\begin{tabular}{|l|l|l|l|l|} 
 \toprule Nome istanza  & Miglior colore & Media & Std  & Tempo di esecuzione \\ 
\midrule
queen5\_5.col  & 5              & 5   & 0 & 24m                 \\ 
queen6\_6.col  & 8              & 8   & 0  & 37m                 \\ 
queen7\_7.col  & 10             & 10.3  & 0 & 54m           \\ 
queen8\_8.col  & --             & 14.2  & 0.39 & 1h and 30m          \\ 
queen8\_12.col & --             & 19.3  & 0.89 & 3h and 12m          \\ 
queen9\_9.col & --             & 17.1  & 1.04 & 2h and 14m          \\ 
DSJC125\_1.col & --             & --  & -- & --      \\ 
DSJC125\_5.col & --             & --  & -- & --      \\ 
DSJC125\_9.col & --             & --  & -- & --      \\ 
le450\_15b.col & --             & --  & -- & --      \\ 
le450\_15c.col & --             & --  & -- & --      \\ 
le450\_15d.col & --             & --  & -- & --      \\ 


\bottomrule
\end{tabular}
\caption{\label{table:tabella2} Risultati ottenuti con la vertex mutation}
\end{center}
\end{table}

\noindent Confrontando le due tabelle si può notare come nelle istanze più piccole il miglior valore cromatico dei grafi è uguale, tuttavia si osservi che non in tutte le iterazioni (runs) viene ottenuto sempre lo stesso valore. Questo è deducibile dai valori della media. Disegnando invece le istanze più grandi è evidente che il numero di colori presenti in un grafo  (ovviamente considerando sempre una rappresentazione cromatica valida) è più basso rispetto alle istanze rappresentate nella  tabella \ref{table:tabella1}.

\section{Conclusioni}
Dai vari esperimenti e dall' approfondimento di questa famiglia di algoritmi si può notare come risultano fondamentali due caratteristiche :
\begin{itemize}
  \item Il tempo di esecuzione
  \item Il risultato ottenuto
\end{itemize}

\noindent Inoltre è essenziale conoscere ed appronfondire le peculiarità del problema che si sta provando ad affrontare. Con tale conoscenza infatti, è possibile costruire degli operatori ad hoc che hanno il doppio compito di trovare la soluzione migliore e di non bloccarsi in un ottimo locale. 

\end{document}

